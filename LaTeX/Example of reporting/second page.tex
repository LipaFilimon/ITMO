%====================ПРЕАМБУЛА=============================
\documentclass[a4paper,10pt]{article}
\usepackage[warn]{mathtext}
\usepackage[T1, T2A]{fontenc}
\usepackage[utf8]{inputenc}
\usepackage[russian,english]{babel}
\usepackage{cmap}
\usepackage{amsmath} 
\usepackage[left=1.4cm,right=2.3cm,top=3.1cm,bottom=3cm]{geometry}
%=================КОЛОНТИТУЛ================================
\usepackage{fancyhdr} %загрузим пакет
\pagestyle{fancy} %применим колонтитул
\fancyhead{} %очистим хидер на всякий случай
\fancyhead[LE,RO]{\thepage} %номер страницы слева сверху на четных и справа на нечетных
\fancyhead[CO]{Степени гиперболических функций}
\fancyhead[LO]{2.425} 
\fancyhead[RO]{121} 
\fancyfoot{} %футер будет пустой
%============================================================
\begin{document}       %вторая полоска в колонтитуле
\begin{center}
\vspace*{-15mm}
\noindent
\hrulefill
\end{center}
\vspace*{1mm}
{\bf 2.424}
\begin{enumerate}
%1
\item \mspace{30mu} $\displaystyle\int th^pxdx=-\dfrac{th^{p-1}x}{p-1}+\int th^{p-2}xdx $ $\mspace{170mu}[p\neq1]$ 
 %%2
\item \mspace{30mu} $\displaystyle\int th^{2n+1}xdx=\sum\limits_{k=1}^n\dfrac{(-1)^{k-1}}{2k}\dbinom nk\dfrac{1}{ch^{2k}x}+\ln chx=$ 

$\displaystyle\mspace{138mu}=-\sum\limits_{k=1}^n\dfrac{th^{2n-2k+2}x}{2n-2k+2}+\ln chx$ 
%%3
\item \mspace{30mu} $\displaystyle\int th^{2n}xdx=-\sum\limits_{k=1}^n\dfrac{th^{2n-2k+1}x}{2n-2k+1}+x$  $\mspace{429mu}Гр1 (351)(12)$ 
%%4
\item \mspace{30mu} $\displaystyle\int cth^{p}xdx=-\dfrac{cth^{p-1}x}{p-1}+\int cth^{p-2}xdx$} $\mspace{147mu}[p\neq1]$
%%5
\item \mspace{30mu} $\displaystyle\int cth^{2n+1}xdx=-\sum\limits_{k=1}^n\frac{1}{2n}\dbinom nk\dfrac{1}{sh^{2k}x}+\ln shx=$ 

$\displaystyle\mspace{145mu} =-\sum\limits_{k=1}^n\dfrac{cth^{2n-2k+2}x}{2n-2k+2}+\ln shx$ 
%%6
\item \mspace{30mu} $\displaystyle\int cth^{2n}xdx=-\sum\limits_{k=1}^n\dfrac{cth^{2n-2k+1}x}{2n-2k+1}+x  $} $\mspace{415mu}Гр1 (351)(14)$
\end{enumerate}


\begin{flushleft}
 Формулы со степенями thx и cthx, равными n=1,2,3,4,см. {\bf 2.423} 17,{\bf 2.423} 22,{\ bf2.423} 27, {\bf 2.2423} 32, {\bf 2.2423} 33,{\bf 2.2423} 38,{\bf 2.2423} 43,{\bf 2.2423} 48.\vspace{3} 
\end{flushleft}


\begin{flushleft}
 {\bf Степени гиперболических функций и гиперболические функции}
 
 {\bf от линейных функций аргумента}
 \end{flushleft}
 \begin{flushleft}
 {\bf 2.425}
 \end{flushleft}
\begin{enumerate}
\item \mspace{30mu} $\displaystyle\int sh(ax+b)sh(cx+d)dx=\dfrac{1}{2(a+c)}sh[(a+c)x+b+d]-$
 
 $\displaystyle\mspace{260mu}-\dfrac{1}{2(a-c)}sh[(a-c)x+b-d]$
 
$\mspace{500mu} {[}a^2\neq{c^2}{]}$ $\mspace{166mu}Гр1 (352)(2a)$

\item \mspace{30mu} $\displaystyle\int sh(ax+b)ch(cx+d)dx= \dfrac{1}{2(a+c)}ch[(a+c)x+b+d]+$

$\displaystyle\mspace{260mu+}\dfrac{1}{2(a-c)}ch[(a-c)x+b-d]$

$\mspace{500mu}  {[}a^2\neq{c^2}{]}$ $\mspace{166mu}Гр1 (352)(2c)$

\item \mspace{30mu} $\displaystyle\int ch(ax+b)ch(cx+d)dx= \dfrac{1}{2(a+c)}sh[(a+c)x+b+d]+$

 $\displaystyle\mspace{260mu+}\dfrac{1}{2(a-c)}sh[(a-c)x+b-d]$

$\mspace{500mu}  {[}a^2\neq{c^2}{]}$ $\mspace{166mu}Гр1 (352)(2b)$
\end{enumerate}

 






\end{document}
%\vspace{5} ОТСТУП !!!!!
%\noindent   отмена красной строки
%\mspace{число} отступ строки
%$$\int ^0_1 x^2 dx$$