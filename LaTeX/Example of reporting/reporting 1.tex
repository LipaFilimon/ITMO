\documentclass[a4paper,12pt]{article}
\usepackage[warn]{mathtext}
\usepackage[T1, T2A]{fontenc}
\usepackage[utf8]{inputenc}
\usepackage[russian,english]{babel}
\usepackage{cmap}
\usepackage{amsmath} 
\usepackage{alltt}
\usepackage[left=2cm,right=2cm,top=2cm,bottom=2cm]{geometry}
\usepackage{moreverb}
\renewcommand\contentsname{Содержание} 
\begin{document}
\selectlanguage{russian}
\begin{titlepage}
\newpage
\begin{center}
\large Санкт-Петербургский политехнический университет Петра Великого\\ \hrulefill
\end{center}
\vspace{8em}
\begin{center}
\huge Отчёт по лабораторной работе\\ [10pt]
\end{center}
\vspace{2.5em}
\begin{center}
\large <<Редактирование учебно-научного текста в системе вёрстки \TeX >>\\
\end{center}

\vspace{6em}

\begin{flushleft}
\emph{Выполнил:} студент 1 курса 13633/1 группы Липинский Илья.\\
\vspace{2.5em}
\emph{Принял:}  \\
\end{flushleft}

\vspace{\fill}

\begin{center}
Санкт-Петербург 2017
\end{center}

\end{titlepage}
\newpage
\setcounter{page}{2} 
%нумерация страниц \setcounter{page}{n}
%\section*{\large Реферат}

\begin{flushleft}
{\LARGE \bf Реферат}
\end{flushleft}

\par\medskip
   Преподавателем были выданы листки с заданием (скан учебно-научного текста), задача заключалась в том, чтобы в системе вёрстки \TeX набрать содержимое этих листов. Студенты, благодаря лекциям и учебной литературе, должны были изучить язык, синтаксис системы и выполнить лабораторную работу.
\par\medskip В результате проделанной работы студент должен овладеть базовыми навыками работы в среде \TeX, также он должен сдать преподавателю отчет по лабораторной работе вместе с скопированными и исходными листами. 
\par\medskip Эта работа будет полезна в образовательных целях всем студентам, так как навык владения системой вёрстки \TeX пригодится в написании научных статей, отчётов.
\par\medskip Данный отчёт состоит из 17 листов, включая: 2а исходных листа и 2е копии 

\newpage
\renewcommand\contentsname{Содержание} 
\tableofcontents
\setcounter{page}{3}

\newpage

\section{Введение}

 В лабораторной работе требовалось в системе вёрстки \TeX скопировать две страницы учебно-научного текста, выданные преподавателем, затем составить отчёт о проделанной работе, учтя ГОСТ 7.32-2001 (Структура и правила оформления отчетов о научно-ис\-сле\-до\-ва\-тель\-ской работе). Сложность заключалась в том, что прежде чем выполнять задание, требовалось изучить язык, синтаксис системы, после этого уже начать вёрстку лабораторной работы. \TeX -очень удобная система вёрстки текстовых документов. В учебное время студенты выполняют различные работы: лабораторные, самостоятельные, пишут статьи, делают презентации, отчёты. Все эти работы удобно реализовывать в \TeX, так как в итоге документ получается читабельным, структурированным и, в целом, хорошо оформленным. Исходя из этих убеждений, владение системной вёрстки \TeX - актуальный навык для студента.

\section{Основная часть}
   Перед студентом ставились следующие задачи:
\begin{enumerate}
\item Скачать и установить дистрибутив и редактор \TeX.
\item Изучить язык, синтаксис системы.
\item Выполнить лабораторную работу.
\item Написать отчёт о проделанной работе.
\end{enumerate}
Рассмотрим установку дистрибутива и редактора в OC Windows 7. Для начала нам надо установить MiKTeX. MiKTeX — открытый дистрибутив TeX для платформы Windows. Одним из существенных достоинств MiKTeX является возможность автоматического об\-но\-вле\-ния установленных компонентов и пакетов. Переходим на официальный сайт miktex.org, далее во вкладку «Downloads», где выбираем подходящий инсталлятор (в моём случае это Basic MiKTeX Installer, 64-bit), скачиваем и начинаем установку. После завершения установки MiKTeX нам нужно скачать и установить редактор \TeX. 

При выполнение работы я использовал Texmaker — кроссплатформенный открытый LaTeX-редактор. Переходим на официальный сайт xm1math.net/texmaker/, далее во вкладку «Downloads», где выбираем подходящий инсталлятор (в моём случае это Texmaker 4.5 for Windows Executable file for windows (xp, vista, seven, 8, 10)), скачиваем и начинаем установку. 
    
Для того, чтобы изучить язык и синтаксис системы, нужно прослушать лекции по данной теме, почитать литературу, а именно: К. В. Воронцов «\LaTeX в примерах», С.М.Львов\-ский «Набор и вёрстка в системе \LaTeX», Tobias Oetiker
Hubert Partl, Irene Hyna и Elisabeth Schlegl «Не очень краткое введение в \ LaTeX или \LaTeX за 137 минут », опробовать и закрепить знания на практике.
После закрепления полученных знаний можно приступать к выполнению лабораторной работы. 

Создадим две папки, где будут находиться документы проектов первой и второй стра\-ницы и еще одну папку для отчёта по лабораторной. Для создания нового документа в программе Texmaker нажимаем кнопку "Создать". В этом документе мы будем набирать первую страницу. Начнём с того, что наберём весь текст, после вписываем формулы и примеры (и графики или рисунки, если требуется). 

\newpage

\begin{flushleft}
{\LARGE \bf Особенности при работе с листами:}
\end{flushleft}
%\vspace{10mm}
\begin{flushleft}
{\Large Текст:}
\end{flushleft}

Для смены шрифтов используются команды вида \verb|\<имя шрифта>|, приведенные в следующей таблице:

\verb|\mathrm| - прямой

\verb|\mathbf| - полужирный

\verb|\mathsf| - рубленый

\verb|\mathit| - курсив

\verb|\mathbb| - для обозначения множеств

Все эти команды действуют на один следующий за ними символ. Если нужно изменить шрифт группы символов, то группу надо заключить в фигурные скобки. 

Имеется восемь размеров шрифта, пронумерованных от 0 до 7 (по умолчанию - 3).
Эти размеры соответствуют следующим директивам LaTeX:

0 - \verb|\tiny| 

1 - \verb|\small|

2 -\verb| \normalsize| 

3 - \verb|\large| 

4 -\verb| \Large|

5 - \verb|\LARGE |

6 - \verb|\huge|

7 - \verb|\Huge| 

Все эти команды изменяют размер шрифт от места появления команды и до конца формулы (или до следующей команды смены размера шрифта). Если нужно изменить размер только части формулы, то нужно писать так \cite{1}\cite{2} :

abcdefg -{\huge abcdefg}- abcdefg

Пример применения команд:

\verb|{\tiny ABCDEFG abcdefg}| \hspace{1cm}{\tiny ABCDEFG abcdefg}   

\verb|{\small ABCDEFG abcdefg}| \hspace{1cm}{\small ABCDEFG abcdefg}  

\verb|{\normalsize ABCDEFG abcdefg}| \hspace{1cm}{\normalsize ABCDEFG abcdefg}  

\verb|{\large ABCDEFG abcdefg| \hspace{1cm}{\large ABCDEFG abcdefg}  

\verb|{\Large ABCDEFG abcdefg}| \hspace{1cm}{\Large ABCDEFG abcdefg}  

\verb|{\LARGE ABCDEFG abcdefg}| \hspace{1cm}{\LARGE ABCDEFG abcdefg}  

\verb|{\huge ABCDEFG abcdefg}| \hspace{1cm}{\huge ABCDEFG abcdefg}  

\verb|{\Huge ABCDEFG abcdefg}| \hspace{1cm}{\Huge ABCDEFG abcdefg}  

\newpage
\begin{flushleft}
{\Large Формулы:}
\end{flushleft}

%\bigskip
Чтобы печатать математические формулы, в преамбуле нужно подключить пакеты:  \verb|amssymb|, \verb|amsmath| и \verb|mathtext|.

LaTeX включает в себя специальный режим для верстки математики.  . Математический текст внутри абзаца вводится между  $\backslash$( и  $\backslash$), между \texttt{\$} и \texttt{\$} или между  \verb|\begin{math}| или \verb|\end{math}|. \cite[3.1 Общие сведения, стр.49]{3} 

При выполнении работы студент может столкнуться с громоздкой и сложной формулой, например , как в данной лабораторной работе:

$\displaystyle\int th^{2n+1}xdx=\sum\limits_{k=1}^n\dfrac{(-1)^{k-1}}{2k}\dbinom nk\dfrac{1}{ch^{2k}x}+\ln chx=$ 

Как же набирать такие формулы ? Для начала мы ставим два символа \texttt{\$}, далее мы будем набирать формулу между ними. 
\begin{description}
\item [Интеграл:] \verb|$\int$|
\hspace{1cm}{$\displaystyle\int$}
\item [Степени:] \verb|$th^{2n+1}$|
\hspace{1cm}{$th^{2n+1}$}
\item [Биномиальныe коэффициенты:] \verb|$C_n^k=\binom nk=\dbinom nk=\tbinom nk$|

\hspace{1cm}{$C_n^k=\binom nk=\dbinom nk=\tbinom nk$}
\item [Сумма (Сигма):] \verb|$\sum\limits_{k=1}^n$|
\hspace{1cm}{$\sum\limits_{k=1}^n$}
\item [Дробь:] \verb|$\dfrac{(-1)^{k-1}}{2k}$|
\hspace{1cm}{$\dfrac{(-1)^{k-1}}{2k}$}
\item [Функции:] Функции типа , имена которых принято набирать прямым шрифтом, набираются с помощью специальных команд, причем команда, как правило, совпадает с именем функции. \cite{2}

\verb| \arg, \cos, \cosh, \cot, \coth, \csc,| 

\verb| \det, \dim, \exp, \gcd, \hom, \inf,| 

\verb| \ker, \lg, \ln, \log, \max, \min,| 

\verb| \sec, \sin, \sinh, \sup, \tan, \tanh,| 

\verb| \arg, \cos, \cosh, \cot, \coth, \csc,| 

\verb| \arccos, \arcsin, \arctan|

$\arg, \cos, \cosh, \cot, \coth, \csc,$

$\det, \dim, \exp, \gcd, \hom, \inf,$

$\ker, \lg, \ln, \log, \max, \min,$
 
$\sec, \sin, \sinh, \sup, \linebreak \tan, \tanh,$

$\arccos, \arcsin, \arctan
$


\end{description}


После того, как мы перенесли всю информацию и данные с исходного листа, мы начинаем редактировать наш документ. В конце концов у нас должна получиться копия исходного листа. Затем повторяем всё то же самое со вторым листом. Когда лабораторная работа будет завершена, приступаем к отчёту. Создаём новый документ и начинаем набирать отчёт по строгому плану:
\begin{enumerate}
\item Титульный лист.
\item Реферат.
\item Содржание.
\item Введение.
\item Основная часть.
\item Заключение.
\item Список литературы.
\item Приложения.
\end{enumerate}
На титульном листе должный быть указаны: организация, в которой выполнена работа, название работы, кто её выполнил, кто её принял, место и год. В реферате студент кратко описывает работу. В содержании указываются все отделы отчёта с номерами страниц. В введении описываются задачи лабораторной, её актуальность, но не указываются решения задач, которые будут написаны в основной части. В заключении студент должен написать то, что он добился в этой работе, описать результаты и оценить затраченное на выполнение лабораторной работы время. В списке литературы указываются литературные или э\-лек\-трон\-ные источники, а в приложении прилепляется исходный код.

\newpage


\section{Вывод}
Чтобы изучить язык, разобраться с синтаксисом системы и "набить руку" мне потребовалось суммарно 12 часов. Чтобы выполнить лабораторную работу,сделать отчёт мне потребовалось около 45 часов. В итоге, я познакомился с системой компьютерной вёрстки TeX. При выполнении работы получил навык работы с этой системой , а так же узанл стандарты написания отчётов и ,собственно, написал свой первый отчёт. Всe полученные знания ,безусловно, пригодятся в будущем.

\newpage
\section{Источники}
\begin{thebibliography}{9}

	\bibitem{1}
		С. М. Львовский,
		\emph{Набор и вёрстка в системе \LaTeX},
		3-е издание,исправленное и дополненное, 2003.

	\bibitem{2}
		К.В. Воронцов,
		\emph{\LaTeX  в примерах},
		16 декабря 2005.
	
	\bibitem{3}
		Tobias Oetiker, Hubert Partl, Irene Hyna,Elisabeth Schlegl,
		\emph{Не очень кратеок введение в \LaTeX  или \LaTeX  за 137 минут },
		Версия 4.12, 13 апреля, 2003. Перевод:Б.Тоботрас, 22 мая 2003 г.

\end{thebibliography}

\newpage
%\section{Приложения}

\begin{flushright}

\section{Приложение 1}
\end{flushright}
\newpage
\setcounter{page}{11} 
\begin{verbatim}
%====================ПРЕАМБУЛА=============================
\documentclass[10pt]{article}
\usepackage[russian,english]{babel}
\usepackage[utf8]{inputenc}
\usepackage[T1,T2A]{fontenc}
\usepackage{amssymb,amsmath,amsfonts,latexsym,mathtext}
\usepackage{indentfirst} %делать отступ в начале параграфа

\usepackage[dvips]{graphics}
\usepackage{pgf}
\usepackage{vmargin}

\usepackage[left=3.4 cm,right=-0.8 cm,top=50mm,bottom=-10 cm]{geometry} 
\usepackage{geometry}
\geometry{paper=a5paper}

%=================КОЛОНТИТУЛ================================
\usepackage{fancyhdr} %загрузим пакет
\pagestyle{fancy} %применим колонтитул
\fancyhead{} %очистим хидер на всякий случай
\fancyhead[C]{2.1. Матрицы и векторы}
\fancyhead[L]{} 
\fancyhead[R]{33} 
\fancyfoot{}%футер будет пустой
%============================================================
\begin{document}
 Часто удобно n-метровый вектор понимать как точку в n-ме-\linebreak рном пространстве. 
 Таким образом, геометрической интерпрета-\linebreak цией вектора является точка .
 В этом смысле вектор-строка и\linebreak вектор-столбец не различаются .
 Используя геометрический язык,\linebreak    
 мы будем поэтому часто векторы называть точками. 
 В двумер-\linebreak ном и трёхмерном пространствах вектор удобно представлять
 \linebreak в виде направленного отрезка , идущего из начала координат
 \linebreak в точку , характеризующую вектор. В этом случае легко могут
 \linebreak быть проиллюстрированы такие операции над векторами , как
 \linebreak умножение на скаляр ( рис. 2.1) и сложение (рис. 2.2).

\begin{figure}[h]
\center{ \includegraphics[width=0.8\linewidth]{scan2}}
\end{figure} 
 \vspace{-3mm}
\par\medskip Чтобы показать,что n-мерное пространство , являющееся\linebreak 
совокупностью всех n-мерных векторов , обладает свойствами \linebreak , 
аналогичными свойствам обычных двумерных и трёхмерных \linebreak пространств , 
введем для этого пространства понятия {\it системы\linebreak координат} и 
{\it расстояния} . Для этого рассмотрим n n-мерных\linebreak векторов\linebreak
$ \textbf{e}_1=[1,0,...,0], \textbf{e}_2=[1,0,...,0],..., \textbf{e}_n=[1,0,...,0],$
\linebreak
называемых {\it единичными векторами} . Тогда любой n-мерный век-\linebreak
 тор может быть записан в виде

 $\mspace{63mu}
  \textbf{a}=[a_1,...,a_n]=a_1\textbf{e}_1+a_2\textbf{e}_2+...+a_n\textbf{e}_n$
   $\mspace{182mu } (2.2)$
\vspace{-6mm}
 \noindent 
 \par\medskip     Говорят ,что вектор \textbf{а} является линейной комбинацией
  дан-\linebreak ных векторов $ \textbf{a}_1,...,\textbf{a}_k $, если существуют 
  такие скаляры $\lambda_j$ , что\linebreak
   $a=\sum\limits_{j=1}^k \lambda_j a_j$. Из равенства (2.2) видно,
   что любой n-мерный век-\linebreak тор может быть представлен как линейная 
комбинация единич-\linebreak ных векторов. Единичные векторы могут быть 
использованы для\linebreak определения системы координат: оси координат в этой системе 
 
\end{document}



\end{verbatim}

\newpage

Копия
\newpage
\begin{flushright}
\section{Приложение 2}
\end{flushright}
\newpage
\setcounter{page}{15} 
\begin{verbatim}%====================ПРЕАМБУЛА=============================
\documentclass[a4paper,10pt]{article}
\usepackage[warn]{mathtext}
\usepackage[T1, T2A]{fontenc}
\usepackage[utf8]{inputenc}
\usepackage[russian,english]{babel}
\usepackage{cmap}
\usepackage{amsmath} 
\usepackage[left=1.4cm,right=2.3cm,top=3.1cm,bottom=3cm]{geometry}
%=================КОЛОНТИТУЛ================================
\usepackage{fancyhdr} %загрузим пакет
\pagestyle{fancy} %применим колонтитул
\fancyhead{} %очистим хидер на всякий случай
\fancyhead[LE,RO]{\thepage} 
\fancyhead[CO]{Степени гиперболических функций}
\fancyhead[LO]{2.425} 
\fancyhead[RO]{121} 
\fancyfoot{} %футер будет пустой
%============================================================
\begin{document}       %вторая полоска в колонтитуле
\begin{center}
\vspace*{-15mm}
\noindent
\hrulefill
\end{center}
\vspace*{1mm}
{\bf 2.424}
\begin{enumerate}
%1
\item \mspace{30mu} $\displaystyle\int th^pxdx=-\dfrac{th^{p-1}x}{p-1}+
\int th^{p-2}xdx $ $\mspace{170mu}[p\neq1]$ 
 %%2
\item \mspace{30mu} $\displaystyle\int th^{2n+1}xdx=\sum\limits_{k=1}^n
\dfrac{(-1)^{k-1}}{2k}\dbinom nk\dfrac{1}{ch^{2k}x}+\ln chx=$ 

$\displaystyle\mspace{138mu}=-\sum\limits_{k=1}^n
\dfrac{th^{2n-2k+2}x}{2n-2k+2}+\ln chx$ 
%%3
\item \mspace{30mu} $\displaystyle\int th^{2n}xdx=-\sum\limits_{k=1}^n
\dfrac{th^{2n-2k+1}x}{2n-2k+1}+x$  $\mspace{429mu}Гр1 (351)(12)$ 
%%4
\item \mspace{30mu} $\displaystyle\int cth^{p}xdx=-\dfrac{cth^{p-1}x}{p-1}+
\int cth^{p-2}xdx$} $\mspace{147mu}[p\neq1]$
%%5
\item \mspace{30mu} $\displaystyle\int cth^{2n+1}xdx=-\sum\limits_{k=1}^n
\frac{1}{2n}\dbinom nk\dfrac{1}{sh^{2k}x}+\ln shx=$ 

$\displaystyle\mspace{145mu} =-\sum\limits_{k=1}^n
\dfrac{cth^{2n-2k+2}x}{2n-2k+2}+\ln shx$ 
%%6
\item \mspace{30mu} $\displaystyle\int cth^{2n}xdx=-\sum\limits_{k=1}^n
\dfrac{cth^{2n-2k+1}x}{2n-2k+1}+x  $} $\mspace{415mu}Гр1 (351)(14)$
\end{enumerate}


\begin{flushleft}
 Формулы со степенями thx и cthx, равными n=1,2,3,4,см. {\bf 2.423} 17,{\bf 2.423} 
 22,{\ bf2.423} 27, {\bf 2.2423} 32, {\bf 2.2423} 33,{\bf 2.2423} 38,{\bf 2.2423}
  43,{\bf 2.2423} 48.\vspace{3} 
\end{flushleft}


\begin{flushleft}
 {\bf Степени гиперболических функций и гиперболические функции}
 
 {\bf от линейных функций аргумента}
 \end{flushleft}
 \begin{flushleft}
 {\bf 2.425}
 \end{flushleft}
\begin{enumerate}
\item \mspace{30mu} $\displaystyle\int sh(ax+b)sh(cx+d)dx=
\dfrac{1}{2(a+c)}sh[(a+c)x+b+d]-$
 
 $\displaystyle\mspace{260mu}-\dfrac{1}{2(a-c)}sh[(a-c)x+b-d]$
 
$\mspace{500mu} {[}a^2\neq{c^2}{]}$ $\mspace{166mu}Гр1 (352)(2a)$

\item \mspace{30mu} $\displaystyle\int sh(ax+b)ch(cx+d)dx
= \dfrac{1}{2(a+c)}ch[(a+c)x+b+d]+$

$\displaystyle\mspace{260mu+}\dfrac{1}{2(a-c)}ch[(a-c)x+b-d]$

$\mspace{500mu}  {[}a^2\neq{c^2}{]}$ $\mspace{166mu}Гр1 (352)(2c)$

\item \mspace{30mu} $\displaystyle\int ch(ax+b)ch(cx+d)dx=
 \dfrac{1}{2(a+c)}sh[(a+c)x+b+d]+$

 $\displaystyle\mspace{260mu+}\dfrac{1}{2(a-c)}sh[(a-c)x+b-d]$

$\mspace{500mu}  {[}a^2\neq{c^2}{]}$ $\mspace{166mu}Гр1 (352)(2b)$
\end{enumerate}

 
\end{document}

\end{verbatim}

\newpage

Копия 

\end{document}